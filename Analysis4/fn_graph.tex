\documentclass[tikz,border=5pt]{standalone}
\usepackage{pgfplots}
\pgfplotsset{compat=1.18}

\begin{document}
\begin{tikzpicture}
    \begin{axis}[
        axis lines = middle,  % Axes pass through the origin
        xmin=0, xmax=1.3,     % Set x-axis limits
        ymin=0, ymax=1.3,     % Set y-axis limits
        xtick={1},            % Only show a tick mark at x=1
        xticklabels={1},    % Label the tick at x=1 as "(1)"
        ytick={1},            % Only show a tick mark at y=1
        yticklabels={1},      % Label the tick at y=1 as "1"
        xlabel=$x$,           % Label for the x-axis
        ylabel=$y$,           % Label for the y-axis
        every axis x label/.style={
            at={(ticklabel* cs:1.05)},
            anchor=west,
        },
        every axis y label/.style={
            at={(ticklabel* cs:1.05)},
            anchor=south,
        },
        width=\linewidth,            % Controls the overall width of the plot
        height=\linewidth,           % Controls the overall height of the plot
    ]

    % Plot for f_1(x) = x
    % We use a domain of 0:1 to show the function only on the interval [0,1]
    \addplot[
        domain=0:1,
        samples=100,
        thick,
        blue
    ] {x} node[pos=0.6, above left] {$f_1(x)$};

    % Plot for f_2(x) = x^2
    \addplot[
        domain=0:1,
        samples=100,
        thick,
        red
    ] {x^2} node[pos=0.6, right] {$f_2(x)$};

    % Plot for f_n(x) = x^n (using n=20 for a steep curve)
    \addplot[
        domain=0:1,
        samples=200, % More samples for a smoother steep curve
        thick,
        green!60!black
    ] {x^20} node[pos=0.8, right] {$f_n(x)$};

    % Add the black dot at the point (1,1)
    \addplot[only marks, mark=*] coordinates {(1,1)};

    \end{axis}
\end{tikzpicture}
\end{document}